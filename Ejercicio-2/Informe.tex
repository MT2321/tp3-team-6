\chapter{Ejercicio 2}
Se desea diseñar una máquina de estados que, al recibir la siguiente secuencia de bits en forma sincrónica 1-1-0-1 encienda una salida y en caso contrario, la mantiene apagada. Se obtienen 5 estados para la misma, en los cuales va a haber un default que va a ser el estado al cual todos los demás estados van a volver en caso de no recibir los deseados además de ser el estado por el cual va a empezar la máquina de estados.\\
Podemos representar los mismos en el siguiente diagrama de estados:\\
\begin{figure}[h!]
	\label{f:Moore}
	\centering
	\includegraphics[scale=0.4]{../Ejercicio-2/Diagrama_de_estados.png}
	\caption{Diagrama de estados}
\end{figure}
En donde Z es la salida dada por la máquina de estados al encontrarse en el estado correspondiente y W es la entrada necesaria para que traicione al siguiente estado y la flecha es la encargada de indicar el sentido de la transición.\\
Este mismo esquema también queda encapsulado en la siguiente tabla de estados:\\
\FloatBarrier
\begin{table}[h!]
	\begin{center}
		\caption{Tabla de estados}
			\begin{tabular}{|c|c c|c|}
			\hline
			\textbf{Estado} &\multicolumn{2}{|c|}{Estado siguiente} & \textbf{Salida}\\
			\textbf{actual} & \textbf{ W=0 } & \textbf{ W=1 } & \textbf{Z}\\
			\hline
			\textbf{Default} & \textbf{ Default } & \textbf{ A } & 0\\
			\hline
			\textbf{A} & \textbf{ Default } & \textbf{ B } & 0\\
			\hline
			\textbf{B} & \textbf{ C } & \textbf{Default } & 0\\
			\hline
			\textbf{C} & \textbf{ Default } & \textbf{ D } & 0\\
			\hline
			\textbf{D} & \textbf{Default } & \textbf{Default} & 1\\
			\hline
			\end{tabular}
	\end{center}
\end{table}
\FloatBarrier
Para la implementación de esté falta realizar la asignación de valores de estado, lo cual nos lleva cambiar la tabla anterior por la siguiente:\\
\FloatBarrier
\begin{table}[h!]
	\begin{center}
		\label{t:Tabla}
		\caption{Tabla de estados asignados}
			\begin{tabular}{|c|c|c c|c|}
			\hline
			\textbf{Estado} & \textbf{Asignacion del} &\multicolumn{2}{|c|}{Estado siguiente} & \textbf{Salida}\\
			\textbf{actual}  & \textbf{Estado actual} & \textbf{ W=0 } & \textbf{ W=1 } & \textbf{Z}\\
			\hline
			\textbf{Default} & 000 & 000 &  001 & 0\\
			\hline
			\textbf{A} & 001 & 000 &  010 & 0\\
			\hline
			\textbf{B} & 010 & 011 & 000 & 0\\
			\hline
			\textbf{C} & 011 & 000 & 100 & 0\\
			\hline
			\textbf{D} & 100 & 000 & 000 & 1\\
			\hline
			\end{tabular}
	\end{center}
\end{table}
\FloatBarrier
%Empieza la parte de la maquina de Moore
\section{Máquina de Moore}
Para nuestro caso tenemos el siguiente circuito secuencial genérico:\\
\FloatBarrier
\begin{figure}[h!]
	\centering
	\includegraphics[scale=0.4]{../Ejercicio-2/Maquina_Moore.png}
	\caption{Circuito generico}
\end{figure}
\FloatBarrier
Donde vamos a utilizar Flip-Flops D dado que la entrada D de estos va a corresponder con el estado siguiente $Y_i$ y van a estar seteados por el clock para que esta salida luego cambia la variable $y_i$ a $Y_i$, dado que $y_i$ son las variables de estado actual.De la tabla \ref{t:Tabla}, obtenemos los siguientes mapas de Karnaugh:\\
\begin{center}
	\begin{figure}[h!]
		\begin{minipage}{0.5\textwidth}
			\caption{Mapa de Karnaugh para $Y_1$}
			\centering
			\begin{Karnaugh}
				\contingut{0,0,1,0,0,X,X,X,1,0,0,0,0,X,X,X}
				\implicant{2}{6}{red}
				\implicant{8}{8}{red}
			\end{Karnaugh}
		\end{minipage}
		 \hspace{5mm}
		\begin{minipage}{0.5\textwidth}
			\caption{Mapa de Karnaugh para $Y_2$}
			\centering
			\begin{Karnaugh}
				\contingut{0,0,1,0,0,X,X,X,0,1,0,0,0,X,X,X}
				\implicant{2}{6}{red}
				\implicant{13}{9}{red}
			\end{Karnaugh}
		\end{minipage}
	\end{figure}
\end{center}
\begin{center}
	\begin{figure}[h!]
		\begin{minipage}{0.5\textwidth}
			\caption{Mapa de Karnaugh para $Y_3$}
			\centering
			\begin{Karnaugh}
				\contingut{0,0,0,0,0,X,X,X,0,0,0,1,0,X,X,X}
				\implicant{15}{11}{red}
			\end{Karnaugh}
		\end{minipage}
		 \hspace{5mm}
		\begin{minipage}{0.5\textwidth}
			\caption{Mapa de Karnaugh para Z}
			\centering
			\begin{Karnaughvuit}
				\contingut{0,0,0,0,1,X,X,X}
				\implicant{4}{6}{red}
			\end{Karnaughvuit}
		\end{minipage}
	\end{figure}
\end{center}
En donde las X representan el don't care y se les decidió dar un valor acorde al cual permiten la simplificación del circuito. Dando como resultado las siguientes ecuaciones:\\
\begin{center}
	$Y_1 = W \cdot \overline{ y_3 } \cdot \overline{ y_2} \cdot \overline{ y_1 } + \overline{ W } \cdot \overline{ y_1} \cdot y_2  $\\
	$Y_2 = W \cdot \overline{ y_2} \cdot  y_1  + \overline{ W } \cdot \overline{ y_1} \cdot y_2  $\\
	$Y_3 = W \cdot  y_2 \cdot  y_1   $\\
	$Z = y_3  $\\
\end{center}
Se realizo la correspondiente simulación en verilog, el cual nos da un comportamiento ideal del circuito, obteniendo el siguiente resultado:\\
\FloatBarrier
\begin{figure}[h!]
	\centering
	\includegraphics[scale=0.8]{../Ejercicio-2/Simulacion_Moore.png}
	\caption{Simulación}
\end{figure}
\FloatBarrier
%Empieza la parte de la maquina de Mealy
\section{Máquina de Mealy}
Para poder realizar la máquina de estados con el modelo de Mealy es necesario que la salida dependa tanto de los estados como de la entrada de esta, con lo cual van a ser necesarios realizar cambios a la actual máquina de estados.\\
Quitando el último estado del diagrama \ref{f:Moore} y expresando la salida junto con la entrada podemos obtener el siguiente diagrama:\\
\FloatBarrier
\begin{figure}[h!]
	\label{f:Meale}
	\centering
	\includegraphics[scale=0.4]{../Ejercicio-2/Diagrama_de_estados_2.png}
	\caption{Diagrama de estados}
\end{figure}
\FloatBarrier
El diagrama tiene un estado menos ahorra, esto provoca los siguientes cambio en la tabla de asignación:\\
\FloatBarrier
\begin{table}[h!]
	\begin{center}
		\label{t:Tabla2}
		\caption{Tabla de estados asignados}
			\begin{tabular}{|c|c|c c|c|c|c|}
			\hline
			\textbf{Estado} & \textbf{Asignacion del} &\multicolumn{2}{|c|}{Estado siguiente} & \multicolumn{2}{|c|}{Salida Z}\\
			\textbf{actual}  & \textbf{Estado actual} & \textbf{ W=0 } & \textbf{ W=1 } & \textbf{ W=0 } & \textbf{ W=1 } \\
			\hline
			\textbf{Default} & 00 & 00 &  01 & 0 & 0\\
			\hline
			\textbf{A} & 01 & 00 &  10 & 0 & 0\\
			\hline
			\textbf{B} & 10 & 11 & 00 & 0 & 0\\
			\hline
			\textbf{C} & 11 & 00 & 00 & 0 & 1\\
			\hline
			\end{tabular}
	\end{center}
\end{table}
\FloatBarrier
La cual nos permite obtener los siguientes mapas de Karnaugh:\\
\begin{center}
	\begin{figure}[h!]
		\begin{minipage}{0.3\textwidth}
			\caption{Mapa de Karnaugh para $Y_1$}
			\centering
			\begin{Karnaughvuit}
				\contingut{0,0,1,0,1,0,0,0}
				\implicant{2}{2}{red}
				\implicant{4}{4}{red}
			\end{Karnaughvuit}
		\end{minipage}
		 \hspace{5mm}
		\begin{minipage}{0.3\textwidth}
			\caption{Mapa de Karnaugh para $Y_2$}
			\centering
			\begin{Karnaughvuit}
				\contingut{0,0,1,0,0,1,0,0}
				\implicant{2}{2}{red}
				\implicant{5}{5}{red}
			\end{Karnaughvuit}
		\end{minipage}
 		\hspace{5mm}
		\begin{minipage}{0.3\textwidth}
			\caption{Mapa de Karnaugh para Z}
			\centering
			\begin{Karnaughvuit}
				\contingut{0,0,0,0,0,0,0,1}
				\implicant{7}{7}{red}
			\end{Karnaughvuit}
		\end{minipage}
	\end{figure}
\end{center}
Dando como resultado de las simplificaciones las siguientes ecuaciones:\\
\begin{center}
	$Y_1 = W \cdot \overline{ y_2} \cdot \overline{ y_1 } + \overline{ W } \cdot \overline{ y_1} \cdot y_2  $\\
	$Y_2 = W \cdot \overline{ y_2} \cdot  y_1  + \overline{ W } \cdot \overline{ y_1} \cdot y_2  $\\
	$Z = W \cdot  y_2 \cdot y_1  $\\
\end{center}
Se realizo la correspondiente simulación en verilog, el cual nos da un comportamiento ideal del circuito, obteniendo el siguiente resultado:\\
\FloatBarrier
\begin{figure}[h!]
	\centering
	\includegraphics[scale=0.8]{../Ejercicio-2/Simulacion_Mealy.png}
	\caption{Simulación}
\end{figure}
\FloatBarrier